\documentclass[]{article}
\usepackage[left=1in,top=1in,right=1in,bottom=1in]{geometry}

\usepackage{relsize}

\usepackage{scrextend}

\usepackage[ngerman]{babel}
\usepackage{eurosym}

\usepackage{enumerate}
\usepackage{enumitem}

\newcommand*{\authorfont}{\fontfamily{phv}\selectfont}
\usepackage{lmodern}


  \usepackage[T1]{fontenc}
  \usepackage[utf8]{inputenc}


\usepackage{amsmath}
\usepackage{amsfonts}
\allowdisplaybreaks

\usepackage[wsol,wsolpubl]{optional}

\newcommand{\optb}[2]{\opt{#1}{\textit{Lösung:} #2}}
\newcommand{\sol}[2]{\opt{#1}{#2}}
\newcommand{\solind}[2]{\opt{#1}{\setlength{\leftskip}{2.5em} \textit{Lösung:}\newline #2}}
\newcommand{\ind}[2]{\opt{#1}{\setlength{\leftskip}{2.5em} #2}}


% \setlength{\leftskip}{2.5em}

\usepackage{abstract}
\renewcommand{\abstractname}{}    % clear the title
\renewcommand{\absnamepos}{empty} % originally center

\renewenvironment{abstract}
 {{%
    \setlength{\leftmargin}{0mm}
    \setlength{\rightmargin}{\leftmargin}%
  }%
  \relax}
 {\endlist}

\makeatletter
\def\@maketitle{%
  \newpage
%  \null
%  \vskip 2em%
%  \begin{center}%
  \let \footnote \thanks
    {\fontsize{14}{15}\selectfont \center  \setlength{\parindent}{0pt} \@title \par}%
}
%\fi
\makeatother




\setcounter{secnumdepth}{0}

\usepackage{longtable,booktabs}

\usepackage{graphicx}

\author{\Large M.Sc. Martin C.
Arnold\vspace{0.05in} \newline\normalsize\emph{}   \and \Large M.Sc.
Jens Klenke\vspace{0.05in} \newline\normalsize\emph{}  }

\title{\center Advanced R for Econometricians \\[1em]\smaller{Semester
overview}  }
 


\date{}

\usepackage{titlesec}

\titleformat*{\section}{\normalsize\bfseries}
\titleformat*{\subsection}{\normalsize\itshape}
\titleformat*{\subsubsection}{\normalsize\itshape}
\titleformat*{\paragraph}{\normalsize\itshape}
\titleformat*{\subparagraph}{\normalsize\itshape}


\usepackage{natbib}
\bibliographystyle{plainnat}
\usepackage[strings]{underscore} % protect underscores in most circumstances



\newtheorem{hypothesis}{Hypothesis}
\usepackage{setspace}

\makeatletter
\@ifpackageloaded{hyperref}{}{%
\ifxetex
  \PassOptionsToPackage{hyphens}{url}\usepackage[setpagesize=false, % page size defined by xetex
              unicode=false, % unicode breaks when used with xetex
              xetex]{hyperref}
\else
  \PassOptionsToPackage{hyphens}{url}\usepackage[unicode=true]{hyperref}
\fi
}

\@ifpackageloaded{color}{
    \PassOptionsToPackage{usenames,dvipsnames}{color}
}{%
    \usepackage[usenames,dvipsnames]{color}
}
\makeatother
\hypersetup{breaklinks=true,
            bookmarks=true,
            pdfauthor={M.Sc. Martin C. Arnold () and M.Sc. Jens
Klenke ()},
             pdfkeywords = {},  
            pdftitle={Advanced R for Econometricians: Semester
overview},
            colorlinks=true,
            citecolor=blue,
            urlcolor=blue,
            linkcolor=magenta,
            pdfborder={0 0 0}}
\urlstyle{same}  % don't use monospace font for urls

\usepackage{booktabs}
\usepackage{longtable}
\usepackage{array}
\usepackage{multirow}
\usepackage{wrapfig}
\usepackage{float}
\usepackage{colortbl}
\usepackage{pdflscape}
\usepackage{tabu}
\usepackage{threeparttable}
\usepackage{threeparttablex}
\usepackage[normalem]{ulem}
\usepackage{makecell}
\usepackage{xcolor}


% add tightlist ----------
\providecommand{\tightlist}{%
\setlength{\itemsep}{0pt}\setlength{\parskip}{0pt}}

\newcommand{\var}{\mbox{\sf Var}}
\newcommand{\E}{\mbox{\sf E}}
\newcommand{\cov}{\mbox{\sf Cov}}
\newcommand{\pr}{\mbox{\sf P}}
\newcommand{\corr}{\mbox{\sf Cor}}

\setlength{\parindent}{0pt}

\begin{document}
	
% \pagenumbering{arabic}% resets `page` counter to 1 
%    

% \maketitle

{% \usefont{T1}{pnc}{m}{n}

{

\vskip 40pt\relax \normalsize\fontsize{11}{12}

\begin{minipage}[t]{.49\textwidth}
Chair of Econometrics

M.Sc. Martin C. Arnold \hskip 15pt \emph{\small }   \par M.Sc. Jens
Klenke \hskip 15pt \emph{\small }   
\end{minipage}% 
%
\hfill
%
\begin{minipage}[t]{.49\textwidth}
  \begin{flushright}
    Sommer 2022\\
    University Duisburg-Essen\\
    Department of Economics\\
  \end{flushright}
\end{minipage}

\vspace{20pt}

}

\setlength{\parindent}{0pt}
\thispagestyle{plain}
{\fontsize{10}{10}\selectfont\raggedright 
{\let\newpage\relax\maketitle}
 % title \par  

}


}

\vskip 8.5pt





\vskip 6.5pt


\noindent  \hypertarget{course-schedule}{%
\section{Course Schedule}\label{course-schedule}}

\begin{table}[!h]
\centering
\begin{tabular}{rlll}
\toprule
Week & Date & Topic & Lecturer\\
\midrule
\cellcolor{gray!6}{1} & \cellcolor{gray!6}{2022-04-04} & \cellcolor{gray!6}{Introduction} & \cellcolor{gray!6}{Jens}\\
2 & 2022-04-11 & R Markdown/Git & Jens\\
\cellcolor{gray!6}{3} & \cellcolor{gray!6}{2022-04-18} & \cellcolor{gray!6}{Easter Monday (self-study topics)} & \cellcolor{gray!6}{}\\
4 & 2022-04-25 & Git & Jens\\
\cellcolor{gray!6}{5} & \cellcolor{gray!6}{2022-05-02} & \cellcolor{gray!6}{ggplot2} & \cellcolor{gray!6}{Jens}\\
6 & 2022-05-09 & dplyr/databases & Jens\\
\cellcolor{gray!6}{7} & \cellcolor{gray!6}{2022-05-16} & \cellcolor{gray!6}{Web Scraping} & \cellcolor{gray!6}{Jens}\\
8 & 2022-05-23 & Advanced Concepts & Martin\\
\cellcolor{gray!6}{9} & \cellcolor{gray!6}{2022-05-30} & \cellcolor{gray!6}{Functional} & \cellcolor{gray!6}{Martin}\\
10 & 2022-06-06 & Whit Monday  (self-study topics) & \\
\cellcolor{gray!6}{11} & \cellcolor{gray!6}{2022-06-13} & \cellcolor{gray!6}{OOP} & \cellcolor{gray!6}{Martin}\\
12 & 2022-06-20 & Profiling /Benchmarking & Martin\\
\cellcolor{gray!6}{13} & \cellcolor{gray!6}{2022-06-27} & \cellcolor{gray!6}{Improving Performance} & \cellcolor{gray!6}{Martin}\\
14 & 2022-07-04 & Rcpp & Martin\\
\cellcolor{gray!6}{15} & \cellcolor{gray!6}{2022-07-11} & \cellcolor{gray!6}{RcppAramdillo} & \cellcolor{gray!6}{Martin}\\
\bottomrule
\multicolumn{4}{l}{\textsuperscript{1} We still reserve the right to modify the schedule if necessary.}\\
\end{tabular}
\end{table}

\hypertarget{assignments}{%
\section{Assignments}\label{assignments}}

In general, there will be two assignments, each of which will determine
10\% of the final grade. \textbf{We reserve the right to change the
schedule} so that we can respond to the development of the course.
However, you will be informed about any possible changes in due time in
Moodle and/or in class.

\begin{itemize}
\item
  The first assignment and detailed information will be handed out on
  \textbf{16\(^{th}\)May}, and the submission deadline will be
  \textbf{30\(^{th}\)May}.
\item
  The second assignment and detailed information will be handed out on
  \textbf{13\(^{th}\)June}, and the submission deadline will be
  \textbf{27\(^{th}\)June}.
\end{itemize}

\hypertarget{final-group-project}{%
\section{Final Group Project}\label{final-group-project}}

The topic list for the final group projects will be available on Moodle
on \textbf{10\(^{th}\)June}. We encourage you to develop your own ideas
for your group project. Please come forward with your suggestions as
early as possible so that we can discuss them. You have time to choose
your final topic until \textbf{17\(^{th}\)June}. For this purpose, a
tool will be set up on Moodle. You must submit your final project by
\textbf{9\(^{th}\)September}. The presentation of the final projects
will take place in the week from \textbf{4\(^{th}\) -- 7\(^{th}\)
October}.

If you have any further question do not hesitate to ask.




\newpage
\singlespacing 
\end{document}
