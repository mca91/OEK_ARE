\documentclass[]{article}
\usepackage[left=1in,top=1in,right=1in,bottom=1in]{geometry}

\usepackage{relsize}

\usepackage{scrextend}

\usepackage[ngerman]{babel}
\usepackage{eurosym}

\usepackage{enumerate}
\usepackage{enumitem}

\newcommand*{\authorfont}{\fontfamily{phv}\selectfont}
\usepackage{lmodern}


  \usepackage[T1]{fontenc}
  \usepackage[utf8]{inputenc}


\usepackage{amsmath}
\usepackage{amsfonts}
\allowdisplaybreaks

\usepackage[wsol,wsolpubl]{optional}

\newcommand{\optb}[2]{\opt{#1}{\textit{Lösung:} #2}}
\newcommand{\sol}[2]{\opt{#1}{#2}}
\newcommand{\solind}[2]{\opt{#1}{\setlength{\leftskip}{2.5em} \textit{Lösung:}\newline #2}}
\newcommand{\ind}[2]{\opt{#1}{\setlength{\leftskip}{2.5em} #2}}


% \setlength{\leftskip}{2.5em}

\usepackage{abstract}
\renewcommand{\abstractname}{}    % clear the title
\renewcommand{\absnamepos}{empty} % originally center

\renewenvironment{abstract}
 {{%
    \setlength{\leftmargin}{0mm}
    \setlength{\rightmargin}{\leftmargin}%
  }%
  \relax}
 {\endlist}

\makeatletter
\def\@maketitle{%
  \newpage
%  \null
%  \vskip 2em%
%  \begin{center}%
  \let \footnote \thanks
    {\fontsize{14}{15}\selectfont \center  \setlength{\parindent}{0pt} \@title \par}%
}
%\fi
\makeatother




\setcounter{secnumdepth}{0}

\usepackage{longtable,booktabs}

\usepackage{graphicx}

\author{\Large M.Sc. Martin C.
Arnold\vspace{0.05in} \newline\normalsize\emph{}   \and \Large M.Sc.
Jens Klenke\vspace{0.05in} \newline\normalsize\emph{}  }

\title{\center Advanced R for Econometricians \\[1em]\smaller{Semester
overview}  }
 


\date{}

\usepackage{titlesec}

\titleformat*{\section}{\normalsize\bfseries}
\titleformat*{\subsection}{\normalsize\itshape}
\titleformat*{\subsubsection}{\normalsize\itshape}
\titleformat*{\paragraph}{\normalsize\itshape}
\titleformat*{\subparagraph}{\normalsize\itshape}


\usepackage{natbib}
\bibliographystyle{plainnat}
\usepackage[strings]{underscore} % protect underscores in most circumstances



\newtheorem{hypothesis}{Hypothesis}
\usepackage{setspace}

\makeatletter
\@ifpackageloaded{hyperref}{}{%
\ifxetex
  \PassOptionsToPackage{hyphens}{url}\usepackage[setpagesize=false, % page size defined by xetex
              unicode=false, % unicode breaks when used with xetex
              xetex]{hyperref}
\else
  \PassOptionsToPackage{hyphens}{url}\usepackage[unicode=true]{hyperref}
\fi
}

\@ifpackageloaded{color}{
    \PassOptionsToPackage{usenames,dvipsnames}{color}
}{%
    \usepackage[usenames,dvipsnames]{color}
}
\makeatother
\hypersetup{breaklinks=true,
            bookmarks=true,
            pdfauthor={M.Sc. Martin C. Arnold () and M.Sc. Jens
Klenke ()},
             pdfkeywords = {},  
            pdftitle={Advanced R for Econometricians: Semester
overview},
            colorlinks=true,
            citecolor=blue,
            urlcolor=blue,
            linkcolor=magenta,
            pdfborder={0 0 0}}
\urlstyle{same}  % don't use monospace font for urls

\usepackage{booktabs}
\usepackage{longtable}
\usepackage{array}
\usepackage{multirow}
\usepackage{wrapfig}
\usepackage{float}
\usepackage{colortbl}
\usepackage{pdflscape}
\usepackage{tabu}
\usepackage{threeparttable}
\usepackage{threeparttablex}
\usepackage[normalem]{ulem}
\usepackage{makecell}
\usepackage{xcolor}


% add tightlist ----------
\providecommand{\tightlist}{%
\setlength{\itemsep}{0pt}\setlength{\parskip}{0pt}}

\newcommand{\var}{\mbox{\sf Var}}
\newcommand{\E}{\mbox{\sf E}}
\newcommand{\cov}{\mbox{\sf Cov}}
\newcommand{\pr}{\mbox{\sf P}}
\newcommand{\corr}{\mbox{\sf Cor}}

\setlength{\parindent}{0pt}

\begin{document}
	
% \pagenumbering{arabic}% resets `page` counter to 1 
%    

% \maketitle

{% \usefont{T1}{pnc}{m}{n}

{

\vskip 40pt\relax \normalsize\fontsize{11}{12}

\begin{minipage}[t]{.49\textwidth}
Chair of Econometrics

M.Sc. Martin C. Arnold \hskip 15pt \emph{\small }   \par M.Sc. Jens
Klenke \hskip 15pt \emph{\small }   
\end{minipage}% 
%
\hfill
%
\begin{minipage}[t]{.49\textwidth}
  \begin{flushright}
    Summer 2022\\
    University Duisburg-Essen\\
    Department of Economics\\
  \end{flushright}
\end{minipage}

\vspace{20pt}

}

\setlength{\parindent}{0pt}
\thispagestyle{plain}
{\fontsize{10}{10}\selectfont\raggedright 
{\let\newpage\relax\maketitle}
 % title \par  

}


}

\vskip 8.5pt





\vskip 6.5pt


\noindent  \hypertarget{course-schedule}{%
\section{Course Schedule}\label{course-schedule}}

\begin{table}[!h]
\centering
\begin{tabular}{rlll}
\toprule
week & date & topic & lecturer\\
\midrule
\cellcolor{gray!6}{1} & \cellcolor{gray!6}{2022-04-04} & \cellcolor{gray!6}{introduction} & \cellcolor{gray!6}{Jens Klenke}\\
2 & 2022-04-11 & rmarkdown/Git & Jens Klenke\\
\cellcolor{gray!6}{3} & \cellcolor{gray!6}{2022-04-18} & \cellcolor{gray!6}{Easter Monday (self-study topics)} & \cellcolor{gray!6}{no class}\\
4 & 2022-04-25 & Git & Jens Klenke\\
\cellcolor{gray!6}{5} & \cellcolor{gray!6}{2022-05-02} & \cellcolor{gray!6}{ggplot2} & \cellcolor{gray!6}{Jens Klenke}\\
6 & 2022-05-09 & dplyr/databases & Jens Klenke\\
\cellcolor{gray!6}{7} & \cellcolor{gray!6}{2022-05-16} & \cellcolor{gray!6}{web scraping} & \cellcolor{gray!6}{Jens Klenke}\\
8 & 2022-05-23 & advanced R concepts & Martin Arnold\\
\cellcolor{gray!6}{9} & \cellcolor{gray!6}{2022-05-30} & \cellcolor{gray!6}{functional programming} & \cellcolor{gray!6}{Martin Arnold}\\
10 & 2022-06-06 & Whit Monday  (self-study topics) & no class\\
\cellcolor{gray!6}{11} & \cellcolor{gray!6}{2022-06-13} & \cellcolor{gray!6}{object oriented programming} & \cellcolor{gray!6}{Martin Arnold}\\
12 & 2022-06-20 & profiling/benchmarking & Martin Arnold\\
\cellcolor{gray!6}{13} & \cellcolor{gray!6}{2022-06-27} & \cellcolor{gray!6}{improving performance} & \cellcolor{gray!6}{Martin Arnold}\\
14 & 2022-07-04 & Rcpp & Martin Arnold\\
\cellcolor{gray!6}{15} & \cellcolor{gray!6}{2022-07-11} & \cellcolor{gray!6}{RcppAramdillo} & \cellcolor{gray!6}{Martin Arnold}\\
\bottomrule
\multicolumn{4}{l}{\textsuperscript{1} We still reserve the right to modify the schedule if necessary.}\\
\end{tabular}
\end{table}

\hypertarget{assignments}{%
\section{Assignments}\label{assignments}}

\textbf{Please note that the schedule is tentative: assignments may be
postponed with respect to course progress. Given so, you will be
informed about any changes in due time via Moodle and in class.}

\begin{itemize}
\item
  There will be two assignments, each contributing 10\% of the final
  grade.
\item
  The first assignment will be issued on \textbf{May 16\(^{th}\)}. The
  submission deadline is \textbf{May 30\(^{th}\)}.
\item
  The second assignment will be issued on \textbf{June 13\(^{th}\)}. The
  submission deadline is \textbf{June 27\(^{th}\)}.
\end{itemize}

\hypertarget{final-group-project}{%
\section{Final Group Project}\label{final-group-project}}

\begin{itemize}
\item
  The final projects should deal with the topics discussed in class but
  may go beyond that. Since this is a programming course, you
  should---regardless of the specific topic of your
  project---demonstrate proficiency in the main paradigms taught (tidy /
  OOP / FP / efficient programming), if applicable.
\item
  You may work in groups of up to 3 students. We expect you to work
  collaboratively, preferably via Git. The final project code should be
  made available in private repositories (more on this later). You
  should specify the contributions of each group member.
\item
  The final project is to be documented in the style of a seminar paper
  / reproducible research report of approx. 30 pages. We will provide a
  rmarkdown template from which you will benefit the most if you pay
  attention to Jens' rmarkdown lesson :-). You must submit your final
  project and the report by \textbf{September 9\(^{th}\)}.
\item
  Projects will be presented in oral presentations (approx. 20 minutes).
  The presentation will be held in the week from \textbf{October
  4\(^{th}\) -- October 7\(^{th}\)}. Further info follows.
\end{itemize}

\hypertarget{grading}{%
\section{Grading}\label{grading}}

The final grade will be the weighted average of the two assignments
(20\% each), the group project (50\%) and the presentations (30\%).

\hypertarget{topics}{%
\section{Topics}\label{topics}}

\begin{itemize}
\item
  Topic proposals will be posted on Moodle on \textbf{June 10\(^{th}\)}.
\item
  We encourage you to propose your own project ideas. Please come
  forward with your suggestions as early as possible so that we can
  discuss them and give you timely feedback.
\item
  Projects should be registered with us until \textbf{June 17\(^{th}\)}.
  We will set up a form on Moodle.
\end{itemize}

Please do not hesitate to ask if you have further questions.




\newpage
\singlespacing 
\end{document}
